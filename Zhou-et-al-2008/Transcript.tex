\documentclass[a5j]{ltjtarticle}

\newcommand*{\titleContent}{『\ruby{新|編}{しん|ぺん}日本語』の\ruby{文|字|起|こし}{も|じ|お|}}
\newcommand*{\authorContent}{趙磊}

\title{\titleContent}
\author{\authorContent}

\usepackage{luatexja-ruby}
\ltjsetruby{stretch=101}

\usepackage{setspace}
\usepackage{enumitem}
\newlist{dialogue}{itemize}{1}
\setlist[dialogue]{label={},align=left,leftmargin=*,labelsep=\zw,nosep}
\newcommand*{\Rospeaks}{\item[魯]}
\newcommand*{\Lispeaks}{\item[李]}
\newcommand*{\Kospeaks}{\item[顧]}

\begin{document}
\maketitle

\section*{第一課 五十音図}

\subsection*{発音練習}

\begin{onehalfspace}
  \ruby{会|う}{あ|}、
  \ruby{家}{いえ}、
  \ruby{上}{うえ}、
  \ruby{絵}{え}、
  \ruby{甥}{おい}、
  \ruby{顔}{かお}、
  \ruby{聞|く}{き|}、
  \ruby{杭}{くい}、
  \ruby{毛}{け}、
  \ruby{\textgt{声}}{こえ}、
  \ruby{酒}{さけ}、
  \ruby{試|合}{し|あい}、
  \ruby{少|し}{すこ|}、
  \ruby{世|界}{せ|かい}、
  \ruby{そ|こ}{其|処}、
  \ruby{高|い}{たか|}、
  \ruby{近|い}{ちか|}、
  \ruby{机}{つくえ}、
  \ruby{手}{て}、
  \ruby{\textgt{年}}{とし}、
  \ruby{中}{なか}、
  \ruby{肉}{にく}、
  \ruby{\textgt{布}}{ぬの}、
  \ruby{猫}{ねこ}、
  \ruby{残|す}{のこ|}、
  \ruby{箸}{はし}、
  \ruby{人}{ひと}、
  \ruby{船}{ふね}、
  \ruby{下手}{へた}、
  \ruby{\textgt{星}}{ほし}、
  \ruby{前}{まえ}、
  \ruby{耳}{みみ}、
  \ruby{昔}{むかし}、
  \ruby{目}{め}、
  \ruby{物}{もの}、
  \ruby{山}{やま}、
  \ruby{雪}{ゆき}、
  \ruby{良|い}{よ|}、
  \ruby{\textgt{楽}}{\textgt{ら}く}、
  \ruby{理|解}{り|かい}、
  \ruby{留守}{るす}、
  \ruby{歴|史}{れ\textgt{き}|し}、
  \ruby{六}{ろく}、
  \ruby{私}{わたし}、
  \ruby{悪|い}{わる|}、
  \ruby{寝|室}{しん|しつ}、
  \ruby{日|本}{に|ほん}、
  \ruby{布|団}{ふ|とん}、
  \ruby{本}{ほん}、
  \ruby{万|年|筆}{まん|ねん|ひつ}、
  アクセント、
  インク、
  クラス、
  \textgt{タオル}、
  テキスト、
  トイレ、
  ナイフ、
  ハンカチ、
  ホテル、
  \ruby{レ|モン}{檸|檬}。
\end{onehalfspace}

\section*{第二課 はじめまして}

\subsection*{濁音練習}

\begin{onehalfspace}
  \ruby{外|国}{がい|こく}、
  \ruby{\textgt{技}|\textgt{師}}{\textgt{ぎ}|し}、
  \ruby{\textgt{具}|\textgt{合}}{ぐ|あい}、
  \ruby{元|気}{げん|き}、
  \ruby{ご|飯}{|はん}、
  \ruby{一|月}{いち|がつ}、
  \ruby{\textgt{鍵}}{か\textgt{ぎ}}、
  \ruby{\textgt{家}|\textgt{具}}{か|\textgt{ぐ}}、
  \ruby{ひげ}{\textgt{髭}}、
  \ruby{リン|ゴ}{林|檎}、
  \ruby{座|席}{ざ|せき}、
  \ruby{事|故}{じ|こ}、
  \ruby{地|図}{ち|ず}、
  ゼミ、
  \ruby{俗|語}{ぞく|ご}、
  \ruby{大|学}{だい|がく}、
  \ruby{縮|む}{ちぢ|}、
  \ruby{続|く}{つづ|}、
  \ruby{電|話}{でん|わ}、
  ドア、
  \ruby{\textgt{場}|\textgt{合}}{ば|あい}、
  ビザ、
  \ruby{豚}{ぶた}、
  \ruby{便|利}{べん|り}、
  \ruby{僕}{ぼく}、
  パンダ、
  ピンク、
  プラス、
  ペン、
  ポスト。
\end{onehalfspace}

\subsection*{前文}

私は\ruby{魯}{ろ}です。あなたは\ruby{李}{り}さんです。あの人は\ruby{顧}{こ}さんです。私は日本語\ruby{科}{か}の一年です。李さんは日本語科の二年です。顧さんは日本語科の三年です。これは服です。それも服です。あれは本です。

\subsection*{会話}

\begin{dialogue}
  \Lispeaks すみません。あなたは魯さんですか。
  \Rospeaks はい、私は日本語科一年の魯です。あなたはどなたですか。
  \Lispeaks 私は日本語科二年の李です。
  \Rospeaks はじめまして。
  \Lispeaks はじめまして。
  \Rospeaks よろしくお願いします。
  \Lispeaks よろしくお願いします。
  \Rospeaks あの人はどなたですか。
  \Lispeaks あの人は日本語科三年の顧さんです。顧さん、こちらは魯さんです。
  \Kospeaks はじめまして。
  \Rospeaks はじめまして。
  \Kospeaks よろしくお願いします。
  \Rospeaks こちらこそよろしくお願いします。
  \item[]
  \Lispeaks これはあなたの荷物ですか。
  \Rospeaks はい、それは私の荷物です。
  \Lispeaks これは何ですか。
  \Rospeaks それは服です。
  \Lispeaks それは何ですか。
  \Rospeaks これも服です。
  \Lispeaks 本はどれですか。
  \Rospeaks 本はあれです。
  \Lispeaks では、案内します。
  \Rospeaks お願いします。
\end{dialogue}

\subsection*{単語}

\begin{onehalfspace}
  はじめまして、\ruby{前|文}{ぜん|ぶん}、\ruby{魯}{ろ}、\ruby{あなた}{貴方}、\ruby{李}{り}、\ruby{あの|人}{|ひと}、\ruby{顧}{こ}、\ruby{日|本|語|科}{に|ほん|ご|か}、\ruby{一|年}{いち|ねん}、\ruby{二|年}{に|ねん}、\ruby{三|年}{さん|ねん}、\ruby{こ|れ}{此|}、\ruby{服}{ふく}、\ruby{そ|れ}{其|}、\ruby{あ|れ}{彼|}、\ruby{会|話}{かい|わ}、\ruby{す|みません}{済|}、はい、\ruby{どなた}{何方}、\ruby{よろ}{宜}しくお\ruby{願}{ねが}いします、\ruby{こちら}{此方}、こちらこそ、\ruby{荷|物}{に|もつ}、\ruby{何}{なん}、\ruby{ど|れ}{何|}、では、\ruby{案|内}{あん|ない}します。
\end{onehalfspace}

\section*{第三課 部屋}

\subsection*{長音練習}

\begin{onehalfspace}
  お\ruby{母}{かあ}さん、お\ruby{ばあ}{祖母}さん、お\ruby{じい}{祖父}さん、\ruby{\textgt{小}|\textgt{さい}}{ちい|}、\ruby{兄}{にい}さん、\ruby{\textgt{数}|\textgt{学}}{すう|がく}、\ruby{\textgt{通}|\textgt{訳}}{つう|やく}、\ruby{先|生}{せん|せい}、\ruby{時|計}{と|けい}、\ruby{姉}{ねえ}さん、お\ruby{父}{とう}さん、\ruby{当|番}{とう|ばん}、\ruby{大}{おお}きい、\ruby{遠}{とお}い、ケーキ、\textgt{シーツ}、スープ、スプーン、テーブル、ノート、デパート。
\end{onehalfspace}

\subsection*{促音練習}

\begin{onehalfspace}
  \ruby{合|作}{がっ|さく}、\ruby{熱|心}{ねっ|しん}、\ruby{欠|席}{けっ|せき}、\ruby{日|記}{にっ|き}、はっきり、マッチ、\ruby{切|手}{きっ|て}、\textgt{ポット}、\ruby{学|校}{がっ|こう}、いっぱい、\ruby{切|符}{きっ|\textgt{ぷ}}、コップ。
\end{onehalfspace}

\subsection*{前文}


\end{document}

% Local Variables:
% TeX-engine: luatex
% End:
